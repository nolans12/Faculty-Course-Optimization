\documentclass{article}
\usepackage[margin=1in]{geometry} % Sets all margins to 1 inch
\usepackage{graphicx} % Required for inserting images
\usepackage{amsmath}

\title{Faculty Course Optimization:\\
Algorithm Formulation}
\author{Nolan Stevenson}
\date{} % This removes the date

\begin{document}

\maketitle

\newpage
\section{Introduction}

	The purpose of this document is to outline the mathematical formulation of the faculty course optimization problem.
    The goal of this algorithm is to maximize the overall faculty satisfaction based of their chosen course preferences while obeying the constraints of TC allocations.

    Faculty members were given a survey asking which courses they would like to teach, where they would rank the courses with a preference of 1-5 or 1-8. Where 1 is the most preferred and 5 or 8 is the least preferred.
    For tenure faculty, they were simply asked to rank their courses. For teaching faculty, some courses had options for section amount, such as 1 section or 2 sections.
    Additionally, some of the courses may be team taught (we will call this splitting), and some courses can't based on the course and TC allocation.

    Thus, the algorithm needs to take into account the faculty preferences, but, also the TC data and specifications for the varioius courses.
    

\section{Problem Formulation}

    The problem can be formulated as a optimization problem, consisting of three key components:
    
    \begin{enumerate}
        \item Decision variables ($x$): These represent the possible choices in the optimization problem.
        \item Objective function: This is the function to be maximized or minimized.
        \item Constraints: These are the limitations or requirements that must be satisfied.
    \end{enumerate}
    
    The objective function and constraints are expressed in linear combination of the decision variables, creating a mathematical formulation suitable for a linear program solver. In this case, we will use the Python package PuLP, which can efficiently solve complex optimization problems.
    
    The problem can be written as: \\
    
    \begin{equation}
        \underset{x_{fcs}}{\text{minimize}} \sum_{f} \sum_{c} \sum_{s} c_{fcs} x_{fcs}
        \label{eq:objective}
    \end{equation}
    
    \text{subject to}

    \begin{equation}
        x_{fcs} \in \{0, 1\} 
        \label{eq:binary}
    \end{equation}

    \begin{equation}
        \sum_{c} \sum_{s} TC(x_{fcs}) = TC_f \quad \forall f
        \label{eq:constraint1}
    \end{equation}

    \begin{equation}
        \sum_{f} \sum_{s} TC(x_{fcs}) \leq TC_c \quad \forall c
        \label{eq:constraint2}
    \end{equation}

    Here, $f$ is the index for faculty, $c$ is the index for course, and $s$ is the index for number of sections taught.
    $TC(x_{fcs})$ is the total TC for a given course and section.
    $TC_f$ is the amount of TC a faculty has to teach.
    $TC_c$ is the total amount of TC a course contains.
    This type of problem is called a mixed-integer linear program (MILP) because it has integer decision variables and constraints
     and an objective that are linear functions of the decision variables.

    The decision variables $x_{fcs}$ are binary, as defined in Equation \eqref{eq:binary}. 
    The optimization solvers job is to decide which decision variables are "turned on" (1) and which are "turned off" (0).
    If $x_{fcs} = 1$, then faculty $f$ is assigned to teach $s$ many sections of course $c$.
    If $x_{fcs} = 0$, then faculty $f$ is not assigned to teach course $c$.

    The objective function, given in Equation \eqref{eq:objective}, is to minimize the total cost of faculty-course assignments based on the survey preferences.
    To make the problem fair for all faculty, the cost coefficients, $c_{fcs}$, are calculated so that the cost of assigning 
    a faculty to a course with preference $p + 1$ (a lower preference) is higher than the cost of assigning all other faculty members 
    a preference of $p$. This can be achieved with:

    \begin{equation}
        c_{fcs} = 2^{\log_2(n)(p_{fcs}-1)} 
        \label{eq:cost}
    \end{equation}

    Where $n$ is the total number of faculty members and $p_{fcs}$ is the preference of faculty $f$ for $s$ sections of course $c$.

    The constraints are given in Equations \eqref{eq:constraint1} and \eqref{eq:constraint2}. 
    Equation \eqref{eq:constraint1} ensures that the total TC allocated to each faculty member matches the required TC for the faculty.
    Equation \eqref{eq:constraint2} ensures that the total TC allocated to each course does not exceed the course's TC capacity, this does allow a course to not have all of its TC filled.

\end{document}
